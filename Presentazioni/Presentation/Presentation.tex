%%%%%%%%%%%%%%%%%%%%%%%%%%%%%%%%%%%%%%%%%%%%%%%%%%
%%%%%%%%%%%%%%%%%%%%%%%%%%%%%%%%%%%%%%%%%%%%%%%%%%
%%
%% Based on the "TASS-2018-Presentation" template
%% (https://github.com/blackzaku/TASS-2018-Presentation)
%%
%% Adapted by Jose Pérez Cano, October 2022
%% (joseperez2000@hotmail.es)
%%
%% Last update: 02/10/2022 (English Support)
%%
%%%%%%%%%%%%%%%%%%%%%%%%%%%%%%%%%%%%%%%%%%%%%%%%%%
%%%%%%%%%%%%%%%%%%%%%%%%%%%%%%%%%%%%%%%%%%%%%%%%%%
%%

\documentclass{beamer} 
\usepackage{babel}
\usepackage{caption}
\usepackage{color}
\usepackage{amsmath}
\usepackage{amssymb}

\usepackage{epstopdf}
\usepackage{graphicx}
\usepackage{soul}
\usepackage[utf8]{inputenc}

\usepackage{hyperref}
\usepackage{xstring}
\graphicspath{{./images/}}


\PassOptionsToPackage{unicode}{hyperref}
\PassOptionsToPackage{naturalnames}{hyperref}
% \captionsetup[table]{font=scriptsize}

\usepackage{booktabs}

%%
% load layout
\usepackage{theme}
\setUPCLayout{draft,newlogo}

\newcommand{\nologo}{\setbeamertemplate{logo}{}} 

%%%%%%%%%%%%%%%%%%%%%%%%%%%%%%%%%%%%%%%%%%%%%%%%%%%%%%%%%%%%
% Info %%%%%%%%%%%%%%%%%%%%%%%%%%%%%%%%%%%%%%%%%%%%%%
%%%%%%%%%%%%%%%%%%%%%%%%%%%%%%%%%%%%%%%%%%%%%%%%%%%%%%%%%%%%
	% title
		\title{Bayesian spatio-temporal models for $PM_{10}$ in the Po valley}	
	% author 
    % (In the mandatory argument "{}", separate multiple
    % authors with "\and" - use "\\" for better author name formatting
    % in the title page. In the optional argument "[]" include all
	% author names, with no "\and" or text formatting macros.)
	% Example: 
    %\author[A. Author Albert Einstein]{Anthony Author \and Albert Einstein}
		\author[Abbr]{Authors}
    % Address

	\institute{\textsc{Politecnico di Milano} \\
 
        
        { Arrigoni Francesca, Baracchi Federica, Cantalini Costanza, Ferrara Stefano, Gjyli Eno, Ursino Bruno  \\}
        
        }
	% date
		\presentationDate{10/11/22}
%%%%%%%%%%%%%%%%


\begin{document}
\section{Description of the problem}
% typeset front slides

\typesetFrontSlides


%%%%%%%%%%%%%%%%%%%%%%%%%%%%%%%%%%%%%%%%%%%%%%%
%
%   SECTION 1
%
%%%%%%%%%%%%%%%%%%%%%%%%%%%%%%%%%%%%%%%%%%%%%%%


\begin{frame}{PM_{10}}

Airborn particulate matter (PM) is a complex mixture of particles that varies in size and chemical composition.

For regulatory purposes particles are defined by their diameter measured in microns.



\end{frame}


\begin{frame}{Where do they come from?}

PM may be either directly emitted from sources (primary particles) or formed in the atmosphere through chemical reactions.
PM10 are often emission from:
\begin{itemize}
    \item combustion processes
    \item construction sites
    \item landfills and agriculture
    \item wildfires
    \item pollen
    \item fragments of bacteria
\end{itemize}
    

\end{frame}

\begin{frame}{Health consequences}
$PM_{10}$ particles can be inhaled, and they deposit on the surfaces of the larger airways of the upper region of the lung inducing tissue damage and lung inflammation.\\
Short-term exposures to $PM_{10}$ have been associated primarily with worsening of respiratory diseases leading to hospitalization and emergency department visits.\\
The effects of long-term exposure to $PM_{10}$ are less clear, although several studies suggest a link between long-term  exposure and respiratory mortality. 

\end{frame}

\begin{frame}{Limit values}
For the protection of human health The EU and the OMS have set two limit values for particulate matter:
\begin{tabular}{|c|c|c|}
\hline 
 & Europe & OMS \\ 
\hline 
annual limit & $40 \mu g / m_3$ & $20 \mu g / m_3$ \\ 
\hline 
daily limit & $50 \mu g / m_3$   max 35 days/year & $50 \mu g / m_3$ max 3 days \\
\hline 
\end{tabular} 

\end{frame}

\section{Dataset}
\begin{frame}{Dataset}
    We have the measures of the concentration of $PM_{10}$ taken from n station over the Po valley and from 2014\\
    
    We are going to focus on the stations of Emilia Romagna and the measurements from 2018.
\end{frame}

\section{Objectives of the project}
\begin{frame}{Objectives of the project}

\begin{itemize}
    \item create an ARIMA model that best represent the problem
    \item make predictions on the short term of the concentration of $PM_{10}$
   
\end{itemize}
    

\end{frame}

%%%%%%%%%%%%%%%%%%%%%%%%%%%%%%%%%%%%%%%%%%%%%%%
%
%   CLOSING
%
%%%%%%%%%%%%%%%%%%%%%%%%%%%%%%%%%%%%%%%%%%%%%%%



\begin{frame}[allowframebreaks]{References}
\bibliographystyle{apalike}
\bibliography{bibliography.bib}
\end{frame}

%%
\end{document}
